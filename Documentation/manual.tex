%\documentclass[english]{article}
\documentclass[a4paper,10pt]{scrartcl}

% \usepackage[T1]{fontenc}
% \usepackage[latin9]{inputenc}
\usepackage{listings}
\usepackage{geometry}
\geometry{verbose,tmargin=2cm,bmargin=2.5cm,rmargin=2.5cm}
% \usepackage{color}
\usepackage[english]{babel}
% \usepackage{url}
% \usepackage[unicode=true,pdfusetitle,
%  bookmarks=true,bookmarksnumbered=false,bookmarksopen=false,
%  breaklinks=true,pdfborder={0 0 0},backref=false,colorlinks=true]
%  {hyperref}
\usepackage[parfill]{parskip}                                                    % no indent, newline for paragraph, better than \setlength\parindent{0pt}     


\usepackage[pdftex,
            pdfauthor={Axel Schild, Vincent Pohl, Gunter Hermann},
            pdftitle={orbkit manual},
            pdfsubject={},
            pdfkeywords={orbkit,atomic orbitals,molecular orbitals,quantum chemistry},
            pdfproducer={Latex},
            pdfcreator={pdflatex},
            colorlinks=true,linkcolor=blue,citecolor=magenta]{hyperref}

\begin{document}

\title{A very short introduction to \texttt{orbkit}}

\maketitle
\texttt{orbkit} is a Python program package
for the generation of molecular orbitals and densities from output
files of MOLPRO, TURBOMOLE, Gamess US, and Gaussian. A list of currently
implemented features (and limitations) can be found in the Overview
to check the suitability of \texttt{orbkit} for your problem. An instruction
for the easy usage of \texttt{orbkit} and the functional principles are described
in the Manual section. The several functions can be separately used
and can be easily implemented in your own program. 

The \texttt{orbkit} support team, Axel, Gunter, and Vincent, welcomes every new
user and will be available to answer your questions. For any change
requests, do not hesitate to contact the \texttt{orbkit} support team.


\section{Overview}

\subsection{Features of \texttt{orbkit}}

\texttt{orbkit} enables the user the calculation of atomic orbitals, molecular
orbitals and densities on an arbitrary regular grid. The grid creator
function generates a user defined grid which can be specified in different
ways. The grid can be centered to a selected atom and the origin. 

By default, orbkit loads the required data from Molden files which
can be created with MOLPRO or TURBOMOLE. Other readable output formats
are Gaussian (.fchk or .log) or Gamess US output files.

The centerpiece of orbkit, the \verb!orbkit_core.py! python module, offers
a variety of calculation options. The user can choose the computation
and the storage of all or just selected molecular orbitals. The storage
of all atomic orbitals is also a possible feature. The density can
be calculated from all or just from several molecular orbitals. The
selection of the molecular orbitals is realized by an external file.
Example files, for the usage of the major features of orbkit, are
available in the program package. 

\texttt{orbkit} can construct different types of output files with the help
of the output Creator (orbkit\_output.py). The output formats include
cube files, HDF5 files, and ZIBAmira Mesh files. \texttt{orbkit} can create
a ZIBAmira network which enables the easy depiction of the densities
or molecular orbitals with ZIBAmira. 

% For an easy handling of orbkit, a graphical version with the most
% important characteristics exists. 

A detailed description of the Python modules of \texttt{orbkit} are listed
in the Function Reference section.


\subsection{Limitations of \texttt{orbkit}}

One limitation of \texttt{orbkit} is the ability to calculate only \emph{s},
\emph{p}, \emph{d}, \emph{f} and \emph{g} atomic orbitals (Molden
file limitation).

For calculation of the atomic orbitals and molecular orbitals,
\texttt{orbkit} requires cartesian gaussians. The usage of spherical gaussian
functions is not yet possible. The usage of cartesian gaussian functions
is standard for the Molden file format, the Gaussian Formatted Checkpoint
File (.fchk), and the Gamess US output. For Gaussian standard output
(.log), this has to be specified in the route section with: \texttt{6D
10F}


\section{Manual}

\texttt{orbkit} can be very easily operated via the terminal. For advanced
users, it is recommended to use \texttt{orbkit} as a Python function. Please
note that the graphical user interface has been removed for the time being
and the option \verb!--no_display! is not necessary.


\subsection{Usage via the Terminal}

For an overview of the available features, a general help text is
shown in the terminal by using the command:

\begin{verbatim}
python orbkit_main.py -h
\end{verbatim}


For the normal calculation of the density from all molecular orbitals
on the standard grid, the following command must be entered in the
console: 

\begin{verbatim}
python orbkit_main.py -i inputfile.molden -o outputfile --no_display
\end{verbatim}


The first two terms start the Python program \verb!orbkit_main.py!. This module
requires an input file defined with \texttt{-i} and the filename and
needs an output file which can be specified by \texttt{-o} followed
by the filename. The file extension of the output will be automatically
created by the program and does not need to be defined. The command
\verb!--no\_display! suppresses the execution of the graphical
interface.

By default, \texttt{orbkit} divides the 3-dimensional grid in 2-dimensional
slices. The atomic orbitals, the molecular orbitals, and the density
are calculated for each slice. At the end of the calculation, the
density is reassembled and stored in an output file. In the default
setting, \texttt{orbkit} performs the density calculation by starting four
subprocesses, which are distributed over the existing CPUs. The number
of subprocesses can be modified with the subsequent command: 
\begin{verbatim}
--numproc=4
\end{verbatim}


For the modification of the grid, one can transfer the modified grid
to the program via an external file and add a special related command
in the terminal: 

\begin{verbatim}
python orbkit_main.py -i inputname -o outputname --read_grid=new_grid.csv --no_display
\end{verbatim}


An example of such an external text file can be found in the program
package and has to be constructed as follows: 

\begin{verbatim}
xmin=,-7.5,xmax=,7.5,Nx=,151
ymin=,-7.5,ymax=,7.5,Ny=,151
zmin=,-7.5,zmax=,7.5,Nz=,151
\end{verbatim}


A special feature of the grid creator is the centering of the grid
to a specific atom and the origin which can be performed by adding
the two commands seen below: 

\begin{verbatim}
-c --atom=A_NUM
\end{verbatim}


\verb!A_NUM! defines the number of the atom to which the grid is centered
and corresponds to the numbering in output file of the respective
quantum chemistry program. 

For the calculation of the density, \texttt{orbkit} takes only into account
the occupied molecular orbitals by default. To perform the density
calculation with a selected set of molecular orbitals, the command: 

\begin{verbatim}
--mo_list=MO_List
\end{verbatim}


can be used. The selection can be specified by an external list \verb!MO_List!
with the structure: 
\begin{verbatim}
1.1 2.1 1.3
1.1 4.1
4.1 2.3 2.1
\end{verbatim}

or 

\begin{verbatim}
1 2 3
3 4
\end{verbatim}

Every row signifies a new calculation of the density from the molecular
orbitals specified in this row. On the one hand, the orbitals can
be defined by the symmetry and on the other, by the order in the respective
output file. In the same manner, the computation and storage of a
selected set of molecular orbitals can be invoked by

\begin{verbatim}
--calc_mo=MO_List
\end{verbatim}


Here, the density computation is omitted.

In \texttt{orbkit}, the output can be issued as standard Gaussian cube files,
HDF5 files or ZIBAmira Mesh files. Additionally, \texttt{orbkit} can directly
create a ZIBAmira network for the easy depiction in ZIBAmira. The
network is called as follows 

\begin{verbatim}
--hx_network
\end{verbatim}


and includes a simple color map creator. The cube file (file extension:
.cb) which is a normal text file and contains the important informations
of the calculation is generated by default. For large systems, the
cube file requires a lot of space on the hard drive and plenty of
time for its creation. In contrast, the HDF5 file format (file extension:
.h5) which is a hierarchical data format can store and organize large
amounts of numerical data. More informations about this file format
can be found at:

\url{http://en.wikipedia.org/wiki/Hierarchical_Data_Format}

Another advantage of this file format is the easy readability with
Matlab or Python. Example files for loading HDF5 files via Matlab
or Python are available in the program package. With the JAVA program,
HDFVIEW, the user can easily load and read the HDF5 files. The command
for the creation of HDF5 files reads as follows

\begin{verbatim}
--hdf5
\end{verbatim}


The last available data format are the ZIBAmira Mesh files (file extension:
.am) which can be directly loaded in ZIBAmira and is generated with
the subsequent command 

\begin{verbatim}
--amira
\end{verbatim}



\subsection{Usage as Python Function}

The first step of the usage of \texttt{orbkit} as a Python function is the
import of the \verb!orbkit_main.py! program as a Python module with the subsequent
Python code line:

\begin{verbatim}
import orbkit_main
\end{verbatim}


An example program code where \texttt{orbkit} is called as a function can
be found in the \verb!orbkit_run_from_fcn.py! file in the program package.
Again, an exemplary calculation of a water molecule with a HDF5 output
file serves for the explanation of the program code.

The name of the input file and the output file have to be defined
with:

\begin{verbatim}
in_fid = 'h2o.md'
out_fid = 'h2o'
\end{verbatim}


After the initialization of \texttt{orbkit} with the following code line:

\begin{verbatim}
orbkit_main.init()
\end{verbatim}


the user can modify the settings for the investigated system. For
the water molecule, the input file, the output file and the output
file format have been selected, and the grid has been modified by
an external file. The following code lines perform these changes:

\begin{verbatim}
orbkit_main.options.hdf5 = True
orbkit_main.options.csv_grid = './Tab.csv'
orbkit_main.options.filename = in_fid
orbkit_main.options.outputname = out_fid
\end{verbatim}
All available features can be found in \verb!orbkit_main.py! in the class cOptions.
In the last step, the user has just to run \texttt{orbkit} as follows

\begin{verbatim}
orbkit_main.main()
\end{verbatim}


Other example files can be found in the \verb!orbkit_module_example.py! file
and \verb!orbkit_derivative_example.py! file in the program package.


\section{Required Python Modules for \texttt{orbkit}}

For a proper execution of \texttt{orbkit}, the following Python modules are
required:


\subsection{General Modules}
\begin{itemize}
\item os - Miscellaneous operating system interfaces (\url{http://docs.python.org/2/library/os.html})
\item sys - System-specific parameters and functions (\url{http://docs.python.org/2/library/sys.html})
\item re - Regular expression operations (\url{http://docs.python.org/2/library/re.html})
\item time - Time access and conversions (\url{http://docs.python.org/2/library/time.html})
\item copy - Shallow and deep copy operations (\url{http://docs.python.org/2/library/copy.html})
\item optparse - Parser for command line options (\url{http://docs.python.org/2/library/optparse.html})
\item csv - CSV File Reading and Writing (\url{http://docs.python.org/2/library/csv.html})
\item multiprocessing - Process-based \textquotedblleft{}threading\textquotedblright{}
interface (\url{http://docs.python.org/2/library/multiprocessing.html})
\end{itemize}

\subsection{Special Modules}
\begin{itemize}
\item scipy - Library of algorithms and mathematical tools (\url{http://en.wikipedia.org/wiki/SciPy})
\item gtk - GTK+ graphical user interface library (\url{http://en.wikipedia.org/wiki/PyGTK})
\item h5py - Interface to the HDF5 binary data format (\url{https://code.google.com/p/h5py/})
\item numpy - Library of high-level mathematical functions (\url{http://en.wikipedia.org/wiki/NumPy})
\end{itemize}

\section{Variables}

\verb!geo_info! $\rightarrow$ only required for creation of output file. python list
(strings); dimensionality: (number of atoms $\times 3$)
\begin{itemize}
\item 1st element: atom symbol
\item 2nd element: atom number (according to output file, i.e., starting
from 1)
\item 3rd element: nuclear charge of atom
\end{itemize}
\verb!geo_spec! $\rightarrow$ python list (floates); dimensionality: (the number of atoms $\times 3$)\\
\verb!ao_spec! $\rightarrow$ python list of python dictionaries (number of dictionaries
corresponds to number of contracted Gaussians):
\begin{itemize}
\item \verb!coeffs!: numpy array, 0th element: coefficient, 1st element: exponent;
dimensionality: (number of primitives $\times 2$)
\item \verb!pnum!: number of primitives (integer)
\item \verb!type!: type of the atomic orbital (string)
\item \verb!atom!: atom number (according to output file, i.e., starting from 1) (integer)
\end{itemize}
\verb!mo_spec! $\rightarrow$ python list of python dictionaries (number of dictionaries
corresponds to number of molecular orbitals):
\begin{itemize}
\item \verb!coeffs!: numpy array (vector); dimensionality: (number of atomic
orbitals $\times 1$)
\item \verb!energy!: energy of molecular orbital (float)
\item \verb!occ\_num!: occupation of molecular orbital (float)
\item \verb!sym!: symmetry of molecular orbital (string)
\end{itemize}
\verb!ao_list! $\rightarrow$ numpy array containing all atomic orbitals on the specified
grid\\
\verb!mo_list! $\rightarrow$ numpy array containing all molecular orbitals on the specified
grid\\
\verb!rho! $\rightarrow$ global numpy array in \verb!orbkit_main.py! containing the density
\end{document}
